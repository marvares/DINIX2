% Options for packages loaded elsewhere
\PassOptionsToPackage{unicode}{hyperref}
\PassOptionsToPackage{hyphens}{url}
%
\documentclass[
]{article}
\usepackage{amsmath,amssymb}
\usepackage{iftex}
\ifPDFTeX
  \usepackage[T1]{fontenc}
  \usepackage[utf8]{inputenc}
  \usepackage{textcomp} % provide euro and other symbols
\else % if luatex or xetex
  \usepackage{unicode-math} % this also loads fontspec
  \defaultfontfeatures{Scale=MatchLowercase}
  \defaultfontfeatures[\rmfamily]{Ligatures=TeX,Scale=1}
\fi
\usepackage{lmodern}
\ifPDFTeX\else
  % xetex/luatex font selection
\fi
% Use upquote if available, for straight quotes in verbatim environments
\IfFileExists{upquote.sty}{\usepackage{upquote}}{}
\IfFileExists{microtype.sty}{% use microtype if available
  \usepackage[]{microtype}
  \UseMicrotypeSet[protrusion]{basicmath} % disable protrusion for tt fonts
}{}
\makeatletter
\@ifundefined{KOMAClassName}{% if non-KOMA class
  \IfFileExists{parskip.sty}{%
    \usepackage{parskip}
  }{% else
    \setlength{\parindent}{0pt}
    \setlength{\parskip}{6pt plus 2pt minus 1pt}}
}{% if KOMA class
  \KOMAoptions{parskip=half}}
\makeatother
\usepackage{xcolor}
\usepackage[margin=1in]{geometry}
\usepackage{longtable,booktabs,array}
\usepackage{calc} % for calculating minipage widths
% Correct order of tables after \paragraph or \subparagraph
\usepackage{etoolbox}
\makeatletter
\patchcmd\longtable{\par}{\if@noskipsec\mbox{}\fi\par}{}{}
\makeatother
% Allow footnotes in longtable head/foot
\IfFileExists{footnotehyper.sty}{\usepackage{footnotehyper}}{\usepackage{footnote}}
\makesavenoteenv{longtable}
\usepackage{graphicx}
\makeatletter
\def\maxwidth{\ifdim\Gin@nat@width>\linewidth\linewidth\else\Gin@nat@width\fi}
\def\maxheight{\ifdim\Gin@nat@height>\textheight\textheight\else\Gin@nat@height\fi}
\makeatother
% Scale images if necessary, so that they will not overflow the page
% margins by default, and it is still possible to overwrite the defaults
% using explicit options in \includegraphics[width, height, ...]{}
\setkeys{Gin}{width=\maxwidth,height=\maxheight,keepaspectratio}
% Set default figure placement to htbp
\makeatletter
\def\fps@figure{htbp}
\makeatother
\setlength{\emergencystretch}{3em} % prevent overfull lines
\providecommand{\tightlist}{%
  \setlength{\itemsep}{0pt}\setlength{\parskip}{0pt}}
\setcounter{secnumdepth}{-\maxdimen} % remove section numbering
\ifLuaTeX
  \usepackage{selnolig}  % disable illegal ligatures
\fi
\usepackage{bookmark}
\IfFileExists{xurl.sty}{\usepackage{xurl}}{} % add URL line breaks if available
\urlstyle{same}
\hypersetup{
  pdftitle={Indices de Bondad de de Ajuste CFA},
  pdfauthor={MVE},
  hidelinks,
  pdfcreator={LaTeX via pandoc}}

\title{Indices de Bondad de de Ajuste CFA}
\author{MVE}
\date{2025-01-02}

\begin{document}
\maketitle

\subsection{Indices}\label{indices}

En el análisis factorial confirmatorio (AFC), especialmente en las
ciencias sociales, se utilizan comúnmente varios índices de bondad de
ajuste para evaluar si el modelo propuesto se ajusta adecuadamente a los
datos. A continuación, te detallo los índices más usados y los niveles
considerados satisfactorios según los estándares académicos:

\subsubsection{\texorpdfstring{\textbf{Índices de Bondad de Ajuste
Comunes}}{Índices de Bondad de Ajuste Comunes}}\label{uxedndices-de-bondad-de-ajuste-comunes}

\begin{enumerate}
\def\labelenumi{\arabic{enumi}.}
\tightlist
\item
  \textbf{Chi-Cuadrado (\(\chi^2\))}:

  \begin{itemize}
  \tightlist
  \item
    Evalúa la discrepancia entre la matriz de covarianza observada y la
    estimada por el modelo.
  \item
    \textbf{Interpretación}:

    \begin{itemize}
    \tightlist
    \item
      Un valor de \(\chi^2\) pequeño y un p-valor \textgreater{} 0.05
      indican buen ajuste.
    \item
      \textbf{Problema}: Es muy sensible al tamaño de muestra; con
      muestras grandes, casi siempre será significativo.
    \end{itemize}
  \item
    \textbf{Uso en práctica}: Se combina con otros índices.
  \end{itemize}
\item
  \textbf{Root Mean Square Error of Approximation (RMSEA)}:

  \begin{itemize}
  \tightlist
  \item
    Mide cuánto se ajusta el modelo a la población en lugar de solo a
    los datos de la muestra.
  \item
    \textbf{Valores de referencia}:

    \begin{itemize}
    \item
      ≤ 0.05: Ajuste excelente.
    \item
      0.05--0.08: Ajuste aceptable.
    \item
      \begin{quote}
      0.10: Mal ajuste.
      \end{quote}
    \end{itemize}
  \end{itemize}
\item
  \textbf{Comparative Fit Index (CFI)}:

  \begin{itemize}
  \tightlist
  \item
    Compara el modelo propuesto con un modelo base (independencia o sin
    correlaciones entre las variables).
  \item
    \textbf{Valores de referencia}:

    \begin{itemize}
    \tightlist
    \item
      ≥ 0.95: Ajuste excelente.
    \item
      ≥ 0.90: Ajuste aceptable.
    \end{itemize}
  \end{itemize}
\item
  \textbf{Tucker-Lewis Index (TLI)} o \textbf{Non-Normed Fit Index
  (NNFI)}:

  \begin{itemize}
  \tightlist
  \item
    Similar al CFI pero penaliza modelos más complejos.
  \item
    \textbf{Valores de referencia}:

    \begin{itemize}
    \tightlist
    \item
      ≥ 0.95: Ajuste excelente.
    \item
      ≥ 0.90: Ajuste aceptable.
    \end{itemize}
  \end{itemize}
\item
  \textbf{Standardized Root Mean Square Residual (SRMR)}:

  \begin{itemize}
  \tightlist
  \item
    Mide la discrepancia promedio entre las correlaciones observadas y
    las estimadas.
  \item
    \textbf{Valores de referencia}:

    \begin{itemize}
    \tightlist
    \item
      ≤ 0.08: Ajuste excelente.
    \item
      ≤ 0.10: Ajuste aceptable.
    \end{itemize}
  \end{itemize}
\item
  \textbf{Goodness-of-Fit Index (GFI)}:

  \begin{itemize}
  \tightlist
  \item
    Evalúa la proporción de varianza-covarianza explicada por el modelo.
  \item
    \textbf{Valores de referencia}:

    \begin{itemize}
    \tightlist
    \item
      ≥ 0.95: Ajuste excelente.
    \item
      ≥ 0.90: Ajuste aceptable.
    \end{itemize}
  \item
    \textbf{Nota}: Este índice ha caído en desuso por críticas
    metodológicas.
  \end{itemize}
\end{enumerate}

\subsubsection{\texorpdfstring{\textbf{Resumen de Niveles de Ajuste
Satisfactorios}}{Resumen de Niveles de Ajuste Satisfactorios}}\label{resumen-de-niveles-de-ajuste-satisfactorios}

\begin{longtable}[]{@{}
  >{\raggedright\arraybackslash}p{(\columnwidth - 4\tabcolsep) * \real{0.2121}}
  >{\raggedright\arraybackslash}p{(\columnwidth - 4\tabcolsep) * \real{0.4091}}
  >{\raggedright\arraybackslash}p{(\columnwidth - 4\tabcolsep) * \real{0.3788}}@{}}
\toprule\noalign{}
\begin{minipage}[b]{\linewidth}\raggedright
\textbf{Índice}
\end{minipage} & \begin{minipage}[b]{\linewidth}\raggedright
\textbf{Nivel Excelente}
\end{minipage} & \begin{minipage}[b]{\linewidth}\raggedright
\textbf{Nivel Aceptable}
\end{minipage} \\
\midrule\noalign{}
\endhead
\bottomrule\noalign{}
\endlastfoot
\(\chi^2\) (p-value) & \textgreater{} 0.05 (no significativo) & No
relevante con muestras grandes \\
RMSEA & ≤ 0.05 & 0.05--0.08 \\
CFI & ≥ 0.95 & ≥ 0.90 \\
TLI/NNFI & ≥ 0.95 & ≥ 0.90 \\
SRMR & ≤ 0.08 & ≤ 0.10 \\
GFI & ≥ 0.95 & ≥ 0.90 \\
\end{longtable}

\subsubsection{Recomendaciones en Ciencias
Sociales}\label{recomendaciones-en-ciencias-sociales}

\begin{itemize}
\tightlist
\item
  \textbf{RMSEA y CFI} son los índices más aceptados y reportados en
  publicaciones académicas.
\item
  El \textbf{SRMR} complementa bien al RMSEA y al CFI porque evalúa las
  discrepancias en las correlaciones.
\item
  Los investigadores suelen reportar múltiples índices para ofrecer una
  visión más completa del ajuste del modelo.
\end{itemize}

\subsubsection{Cálculo en R (Ejemplo)}\label{cuxe1lculo-en-r-ejemplo}

Usando lavaan:

library(lavaan)

\paragraph{Ajustar el modelo}\label{ajustar-el-modelo}

ajuste \textless- cfa(modelo, data = datos)

\paragraph{Índices de ajuste}\label{uxedndices-de-ajuste}

fitmeasures(ajuste, c(``rmsea'', ``cfi'', ``tli'', ``srmr'', ``gfi''))

\end{document}
